\documentclass[11pt,fleqn]{article}
\usepackage[a4paper, hmargin={2.8cm, 2.8cm}, vmargin={2.8cm, 2.8cm}]{geometry}  % Geometri-pakke: Styrer bl.a. maginer                              %
\usepackage[utf8]{inputenc}                                         % Lidt kodning så der ikke kommer problemer ved visse konverteringer            %
\usepackage[babel, lille, nat, da, farve]{ku-forside}               % KU-forside med logoer                                                         %
\usepackage{listingsutf8}
\usepackage{amsmath}
\def\HyperLinks{                                                    %  Hyperlinks-pakke, der laver referencer til links og tillader links til www   %
\usepackage[pdftitle={\TITEL},pdfauthor={\FORFATTER},               %  Der er foretaget et lille trick så pakken indlæses efter                     %
pdfsubject={\UNDERTITEL}, linkbordercolor={0.8 0.8 0.8}]{hyperref}} %  titlen defineres.                                                            %

\usepackage{lastpage}
\usepackage{fancyhdr}
\usepackage[toc]{appendix}
\usepackage{pdfpages}
\usepackage{float}
\usepackage{graphicx}
\usepackage{amsfonts}

\forfatter{Mathias Svensson, Ronni Elken Lindsgaard, Jacob Kirstejn \&
  Philip Munksgaard}
\dato{7 Januar 2011}
\titel{SS Projekt}
\undertitel{Den logaritmiske normalfordeling}

\HyperLinks % Henter hyperlinks-pakke og sætter pdf-titel mm. til at svare til de just definerede

\pagestyle{fancy}
\lhead{{\small \FORFATTER}}
\chead{}
\rhead{{\small \TITEL}}
\cfoot{\footnotesize Side \thepage \ af \pageref{LastPage}}

\renewcommand{\appendixtocname}{Bilag}
\renewcommand{\appendixpagename}{Bilag}
\setcounter{secnumdepth}{1}
\setcounter{tocdepth}{2}
\mathcode`\*"8000{\catcode`\*\active\gdef*{\cdot}}
